\chapter{Trabalhos científicos}

%Capítulo 2 de Szwarcfiter, \textit{Grafos e Algoritmos Computacionais}~\cite{Szwarcfiter1986grafos}.

%%%%%%%%%%%%%%%%%%%%%%%%%%%%%%%%%%%%%%%%%%%%%%%%%%%%%%%%%%%%
\section{Tipos de trabalhos científicos}

\begin{easylist}
& Artigo científico: relatório escrito e publicado descevendo pesquisa original~\cite{Day1998write}
\SKIP
& Monografia: documento que apresenta de forma organizada uma contribuição para o estado da arte. Apresenta informações que não eram conhecidas e que, a partir do momento em que são publicadas, passam a integrar o corpo de conheimento relevante de uma determinada área~\cite{Wazlawick2014metodologia}.
&& TCC: trabalho de graduação que, na prática, pode ser um trabalho técnico, um sistema ou um protótipo. Demonstra que o aluno é capaz de aplicar as técnicas aprendidas durante o curso.
&& Especialização: trabalho de pesquisa ou técnico com grau de exigência semelhante ao TCC.
&& Dissertação de mestrado: trabalho científico que apresenta alguma informação nova sobre um tema relevante para a área.
&& Tese de doutorado: trabalho científico com mesmo formato que a dissertação de mestrado mas com maior dificuldade e profundidade. Deve ser original e substancial~\cite{Comer2008dissertation}. Exige a resolução de um problema mais difícil e mais contribuições signifiativa~\cite{Chinnek1999thesis}.
\end{easylist}

%%%%%%%%%%%%%%%%%%%%%%%%%%%%%%%%%%%%%%%%%%%%%%%%%%%%%%%%%%%%
\section{As partes de um artigo ou trabalho científico}

\begin{easylist}
& IMRaD: \textit{introduction, methods, results and discussion}~\cite{Day1998write}
&& \textit{Introduction}: o que foi estudado?
&& \textit{Methods}: como foi estudado?
&& \textit{Results}: o que foi encontrado?
&& \textit{Discussion}: o que os resultados significam?
\SKIP
& As partes de um artigo ou trabalho científico
&& Introdução: (1) deve apresentar a natureza e motivação do problema sendo investigado. (2) Deve revisar a literatura para orientar o leitor. (3) Deve indicar o método da pesquisa e explicar as razões da escolha do método. (4) Deve apresentar os principais resultados da pesquisa. (5) Deve apresentar as principais conclusões sugeridas pelos resultados. (6) Deve apresentar o escopo, objetivos do trabalho, questão de pesquisa ou hipótese sendo investigada.
&& Trabalhos relacionados: opcionalmente, em trabalhos com menos limitações de espaço como em teses e monografias, pode se adicionar esta seção ou capítulo para permitir uma apresentação mais detalhada da literatura do que na introdução. Nesse caso, a introdução ainda fará uma revisão breve e resumida da literatura.
&& Fundamentação teórica (\textit{background}): também opcional, apresenta conceitos e definições importantes para a compreensão do trabalho. Pode citar referências antigas, clássicas e livros texto. Em trabalhos com espaço mais restrito, pode-se apenas citar as referências onde se podem encontrar os conceitos e definições, economizando espaço.
&& Metodologia / material e métodos: deve apresentar o desenho experimental com detalhes suficientes para permitir a reprodução dos experimentos. Nas ciências biológicas, tipicamente se chama \textit{material e métodos} pois é necessário apresentar os materiais usados em laboratório, reagentes, amimais, plantas, microorganismos, critério de seleção de pacientes ou voluntários humanos, equipamentos etc. Os métodos apresentam as etapas realizadas para se chegar aos resultados. Também apresentam o \textit{dataset} usado, suas características, como foi adquirido ou se foi usado um \textit{dataset} de terceiros. Tipicamente começa com um parágrafo sumarizando as etapas realizadas. Pode apresentar uma figura com um fluxograma das etapas do método. Geralmente é dividida em seções, cada uma com uma etapa do método, mais uma para descrever o \textit{dataset}, preferencialmente logo no início. É excrita predominantemente no pretérito.
&& Discussão: apresenta a interpretação dos resultados, suas implicações teóricas e práticas. Os principais componentes da discussão são os seguintes. (1) Apresenta os princípios, relações e generalizações mostrados pelos resultados. Não é necessário recapitular os resultados. (2) Apresenta as exceções ou faltas de correlação observadas nos dados. Não tente ocultar dados que não se encaixem no que é esperado. (3) Apresenta como os resultados concordam ou discordam de publicações prévias. (4) Discute implicações teóricas e aplicações práticas da pesquisa. (5) Indica as conclusões de forma clara e objetiva. (6) Indica as evidêncais para se chegar a cada conclusão.
&& Conclusões: (1) Resume os principais resultados encontrados. (2) Responde a queistão de pesquisa ou aborda a hipótese levantada. (3) Indica as implicações, significado e conclusões da pesquisa. (4) Encerra apresentando possíveis trabalhos futuros sobre o tema.
\end{easylist}

%%%%%%%%%%%%%%%%%%%%%%%%%%%%%%%%%%%%%%%%%%%%%%%%%%%%%%%%%%%%
\section{Trabalho de Conclusão de Curso no DCOMP-UFS}

\begin{easylist}
& TCC 1
&& Contextualização: contida na introdução, apresenta o tema de forma ampla. Deve contar uma história que leve o leitor de um cenário geral ao tema da pesquisa.
&& Motivação ou justificativa: também contida na introdução, explica porque o trabalho é relevante, mostrando o impacto potencial ou necessidade de investigação.
&& Objetivos: apresentados ao final da introdução geralmente em uma seção específica dividida em objetivo geral e objetivos específicos. Devem ser claros, concisos, específicos e alcançáveis com o tempo e recursos disponíveis.
&& Revisão bibliográfica: capítulo após a introdução, apresenta o que já foi feito, teorizado ou descoberto sore o tema de maneira mais profunda e detalhada do que na contextualização. Estabelece o estado da arte e as lacunas dos estudos anteriores. Fornece a base para justificar a metodologia e para discutir e interpretar os resultados.
&& Plano de continuidade: descreve a metodologia que se pretende usar no TCC 2, ou seja, a técnica a ser usada para resolver o problema e, se houver, o \textit{dataset}. Também apresenta um cronograma de execução.
\SKIP
& TCC 2
&& Possui as mesmas partes do TCC 1 exceto o plano de continuidade.
&& Metodologia: versão refinada e detalhada do plano de continuidade sem o cronograma.
&& Resultados: apresenta os dados obtidos através da metodologua em forma de gráficos e/ou tabelas. Também discute os resultados obtidos e suas implicações.
&& Conclusões: apresenta um sumário das conclusões obtidas através da discussão, contribuições, considerações finais e trabalhos futuros.
\end{easylist}



