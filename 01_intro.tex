\chapter{Metodologia científica}

\setcounter{page}{1}    % set page to 1 again to start arabic count
\pagenumbering{arabic}

%Capítulo 2 de Szwarcfiter, \textit{Grafos e Algoritmos Computacionais}~\cite{Szwarcfiter1986grafos}.

%%%%%%%%%%%%%%%%%%%%%%%%%%%%%%%%%%%%%%%%%%%%%%%%%%%%%%%%%%%%
\section{Ciência e tecnologia}

\begin{easylist}
& Ciência e tecnologia: a Tabela~\ref{tab:cet} compara ciência e tecnologia.
&& Ciência: sistema de acúmulo de conhecimento confiável~\cite{Zobel1997writing}. Conhecimento positivo (empírico e verificável) sistematizado~\cite{Multhauf1959scientist}.
&& Tecnologia: aplicação do conhecimento nas atividades práticas, como atividades industriais e econômicas~\cite{Wazlawick2014metodologia}.
\SKIP
& Método científico~\cite{Marder2011research}
&& Estabelecer uma hipótese
&& Elaborar um procedimento experimental para testar a hipótese
&& Construir instrumentos se necessário
&& Executar os experimentos
&& Analisar os dados obtidos e testar se a hipótese se comprova ou não
\end{easylist}


%%%%%%%%%%%%%%%%%%%%%%%%%%%%%%
\begin{table}[bt]
  \caption{Comparação entre ciência e tecnologia}
  \label{tab:cet}
  \centering
  %\begin{adjustbox}{width=\columnwidth}
    \begin{tabular}{cc}
      \toprule[1.5pt]
      Ciência  & Tecnologia \\
      \midrule
      \makecell{Constrói teorias para\\explicar fatos observados} & \makecell{Aplica o conhecimento\\para transformar o mundo} \\
      \makecell{Caráter teórico,\\analítico} & Caráter prático \\
      \makecell{Resultados teóricos são\\permanentes, duradouros} & Tecnologia é efêmera \\
      \bottomrule[1.5pt]
    \end{tabular}
  %\end{adjustbox}
\end{table}
%%%%%%%%%%%%%%%%%%%%%%%%%%%%%%


%\begin{figure}[b]
%  \begin{center}
%    \begin{tabular}{c}
%      \includegraphics[width=0.7\textwidth]{images/02/fig1.png}
%    \end{tabular}
%  \end{center}
%  \caption{\label{fig:1} Um grafo $G(V,E)$ e sua representação geométrica.}
%  \source{Szwarcfiter~\cite{Szwarcfiter1986grafos}.}
%\end{figure}

\section{Pesquisa}

\begin{easylist}
& Pesquisa: atividade que busca aumentar o conhecimento sobre como o mundo vunciona. Engloba de pesquisas eleitorais a pesquisas científicas~\cite{Wazlawick2014metodologia}.
& Natureza
&& Trabalho original: busca apresentar conhecimento novo a partir de observações e teorias construídas para explicá-las.
&& Resumo de assunto (review): sistematiza uma área de conhecimento, indicando sua evolução histórica, taxonomia e estado da arte.
\SKIP
& Objetivos
&& Exploratória: o autor não tem necessariamente uma hipótese ou objetivo definido em mente. O autor vai examinar um conjunto de fenômenos buscando lacunas que possam ser a base para uma pesquisa mais elaborada.
&& Descritiva: busca obter dados mais consistentes sobre determinado assunto sem tentar criar teorias que expliquem os fenômenos. Caracterizada pelo levantamento de dados ou aplicação de entrevistas e questionários.
&& Explicativa: analisa os dados e busca suas causas e explicações. É a mais complexa e completa.
\SKIP
& Procedimentos técnicos 
&& Bibliográfica: estudo de artigos, teses, livros e publicações indexadas. Não produz conhecimento, apenas supre o pesquisador de informações públicas.
&& Documental: análise de documentos ou dados ainda não publicados. Podem ser examinados relatórios, arquivos, bancos de dados, correspondências etc.
&& Experimental: envolve algum experimento com variáveis experimentais (independente, preditora), controladas pelo pesquisador, e observadas (dependente, resposta), cuja medição e análise pode levar à conclusão de que possuem alguma relação com as variáveis experimentais. Usa técnicas de amostragem e testes de hipóteses.
&& De levantamento (survey): consiste em obter dados através de observações e entrevistas com objetivo de observar uma população.
\end{easylist}

\section{Métodos de pesquisa científica}

\begin{easylist}
& Pesquisa experimental: medir um valor
&& Identificar uma quantidade bem definida
&& Elaborar um procedimento para medi-la
&& Executar os experimentos
&& Analisar os dados obtidos e relatar os resultados
\SKIP
& Pesquisa experimental: medir uma função ou relação
&& Observar um fenômeno e desenvolver questões testáveis
&& Identificar variáveis independentes e dependentes
&& Elaborar um procedimento controlado para variar as variáveis independentes e medir as dependentes, mantendo outros fatores constantes.
&& Executar os experimentos
&& Analisar a relação entre as variáveis e caracterizá-las matematicamente
\end{easylist}


